\documentclass[12pt,a4paper]{article}
\usepackage{url}
\begin{document}
First, for the null hypothesis, we assume complete independence
between the three variables.

If we refer to rows, columns and planes as $r,c,p$, (with dimensions
$R,C,P$) with each cell containing the observed value $o_{rcp}$ then
we can define the total, $t$, for a particular row, $r$, as:
$$t_{r++} = \sum_{c=1}^{C}\sum_{p=1}^{P}o_{rcp}$$
\noindent(where a subscript of $+$ indicates summation over the
appropriate index)

Similarly for columns and planes:
$$t_{+c+} = \sum_{r=1}^{R}\sum_{p=1}^{P}o_{rcp}$$
$$t_{++p} = \sum_{r=1}^{R}\sum_{c=1}^{C}o_{rcp}$$

The expected value for a given cell, $e_{rcp}$ is then:
$$e_{rcp} = \frac{t_{r++} \times t_{+c+} \times t_{++p}}{N^2}$$
\noindent (where $N$ is the total number of observations).

We then calculate the chi-squared value as normal:
$$\chi^2 = \sum_{r=1}^{R}\sum_{c=1}^{C}\sum_{p=1}^{P}\frac{(o_{rcp} - e_{rcp})^2}{e_{rcp}}$$

The number of degrees of freedom, $D$, is simply:
$$D = (R-1)(C-1)(P-1)$$

The calculation of the expected values is based on information at:
\begin{itemize}
\item\url{web.ntpu.edu.tw/~cflin/Teach/Cate/06CateUEN05ThreeWayPPT.pdf}
\item\url{onlinecourses.science.psu.edu/stat504/book/export/html/102}
\item Lienert \& Wolfrun (1980) \emph{Biometrical Journal} {\bfseries
  22}:159--167
(\url{onlinelibrary.wiley.com/doi/10.1002/bimj.4710220209/pdf}).
\end{itemize}

\end{document}
